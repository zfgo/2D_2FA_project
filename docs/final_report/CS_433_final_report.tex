\documentclass[11pt]{article} % EXPAND THE PREAMBLE TO SEE ALL OF THECODE
\usepackage[utf8]{inputenc}
\usepackage{amsmath, amsfonts} % math features and the \begin{aligh*}feature
\usepackage[margin=1.0in]{geometry} % [margin=1.0in]
\usepackage{graphicx} % more arguments for the \includegraphics command
\usepackage{enumitem} % allows better formatting for list environments
\newcommand{\comment}[1]{} % for multiline comments
\usepackage[stretch=10]{microtype} % best package ever
% \usepackage{changepage} \usepackage[symbol]{footmisc} % use symbols
% instead of numbers for footnotes
\usepackage{titletoc} % better toc formatting
\usepackage{titlesec} % to make fancier section/sub/subsubsectionformatting
\usepackage[nottoc]{tocbibind} % add references to the toc
\usepackage{hyperref} % for hyper links
\usepackage{cite} % for citing bibtex

\usepackage{titletoc} % better toc formatting

\title{Final Report: 2D-2FA Software Implementation}
\author{Zane Globus-O'Harra, Doug Ure}
\date{\textit{9 June 2023}}

\begin{document}
\maketitle

\section*{Abstract}

TODO TODO TODO TODO TODO TODO TODO TODO TODO 

\tableofcontents

%%%%%%%%%%%%%%%%%%%%%%%%%%%%%%%%%%%%%%%%%%%%%%%%%%%%%%%%%%%%
%%%%%%%%%%%%%%%%%%%%%%%%%%%%%%%%%%%%%%%%%%%%%%%%%%%%%%%%%%%%
\section{Introduction}

TODO: need to go through and reword/edit this section

Two-factor authentication (2FA) is the contemporary approach to
authorizing a user. The first authentication factor is the user's
password, with the second factor being an additional piece of
information that only the user could know, often a PIN or a push
notification sent to the user over a secure channel to a trusted device. However, both
PIN-2FA and push-based 2FA have some issues, which are addressed in the
paper ``2D-2FA: A New Dimension In Two-Factor Authentication'' by
Maliheh Shirvanian and Shashank
Agrawal \cite{shirvanian2d2fa}.

Some attacks to these two common 2FA methods include shoulder surfing,
short PINs, and neglectful user approvals. The authors present a new
approach to 2FA, which they have coined ``2D-2FA.'' In this new
approach, when a user logs in with their username and password, a
unique identifier is displayed to them. The user then inputs
this same identifier on their device. A one-time PIN is generated on the
device, and transferred automatically to the server, along with the
identifier. The identifier is used in the PIN's computation, so that the
PIN is bound to a specific session. 

The user's device and the server agree on a secret key during a one-time
registration process, which is also used in the PIN computation. Once
the PIN is transferred to the server, the server authenticates the
session associated with the identifier by verifying the PIN, thereby
taking two dimensions into account (the PIN and the identifier).

%%%%%%%%%%%%%%%%%%%%%%%%%%%%%%%%%%%%%%%%%%%%%%%%%%%%%%%%%%%%
\subsection{Keywords}



%%%%%%%%%%%%%%%%%%%%%%%%%%%%%%%%%%%%%%%%%%%%%%%%%%%%%%%%%%%%
\subsection{Motivations}

We are motivated to work on this project for several key reasons.
Firstly, the subject of this project has real-world relevance. 2FA is
becoming increasingly more common to authenticate users, and as it
becomes more prevalent, attackers will focus more of their efforts on
finding ways to break through its layers of security. Our project will
help us learn about ways to further increase the security of 2FA by
using additional information along with the user's credentials and the
server-provided ``identifier.''

This is also a learning opportunity for us. Neither of us are very
familiar with security, and it is something that we are very interested
in learning about. By completing this project, we will develop valuable
technical skills and increase our knowledge base, as well as preparing
us for future projects and industry roles.

Lastly, this project could have a real impact on end-users. 2FA enhances
users' trust and confidence in online systems by making their personal
information and online interactions more secure. This project has the
potential to contribute to a larger goal of make a more secure digital
environment. 

%%%%%%%%%%%%%%%%%%%%%%%%%%%%%%%%%%%%%%%%%%%%%%%%%%%%%%%%%%%%
\subsection{Objectives}

The objectives 

%%%%%%%%%%%%%%%%%%%%%%%%%%%%%%%%%%%%%%%%%%%%%%%%%%%%%%%%%%%%
%%%%%%%%%%%%%%%%%%%%%%%%%%%%%%%%%%%%%%%%%%%%%%%%%%%%%%%%%%%%
\section{Related Work}

In this section, we will look at some traditional 2FA implementations,
chiefly hardware token-based authentication and single sign-on software
token-based authentication.

\subsection{Hardware Token Authentication}



%%%%%%%%%%%%%%%%%%%%%%%%%%%%%%%%%%%%%%%%%%%%%%%%%%%%%%%%%%%%
\subsection{Single Sign-On Software Token Authentication}




%%%%%%%%%%%%%%%%%%%%%%%%%%%%%%%%%%%%%%%%%%%%%%%%%%%%%%%%%%%%
%%%%%%%%%%%%%%%%%%%%%%%%%%%%%%%%%%%%%%%%%%%%%%%%%%%%%%%%%%%%
\section{2D-2FA}



%%%%%%%%%%%%%%%%%%%%%%%%%%%%%%%%%%%%%%%%%%%%%%%%%%%%%%%%%%%%
\subsection{Design}



%%%%%%%%%%%%%%%%%%%%%%%%%%%%%%%%%%%%%%%%%%%%%%%%%%%%%%%%%%%%
\subsection{Implementation}



%%%%%%%%%%%%%%%%%%%%%%%%%%%%%%%%%%%%%%%%%%%%%%%%%%%%%%%%%%%%
%%%%%%%%%%%%%%%%%%%%%%%%%%%%%%%%%%%%%%%%%%%%%%%%%%%%%%%%%%%%
\section{Results}



%%%%%%%%%%%%%%%%%%%%%%%%%%%%%%%%%%%%%%%%%%%%%%%%%%%%%%%%%%%%
%%%%%%%%%%%%%%%%%%%%%%%%%%%%%%%%%%%%%%%%%%%%%%%%%%%%%%%%%%%%
\section{Analysis}



%%%%%%%%%%%%%%%%%%%%%%%%%%%%%%%%%%%%%%%%%%%%%%%%%%%%%%%%%%%%
%%%%%%%%%%%%%%%%%%%%%%%%%%%%%%%%%%%%%%%%%%%%%%%%%%%%%%%%%%%%
\section{Conclusion}



%%%%%%%%%%%%%%%%%%%%%%%%%%%%%%%%%%%%%%%%%%%%%%%%%%%%%%%%%%%%
\subsection{Complications}



%%%%%%%%%%%%%%%%%%%%%%%%%%%%%%%%%%%%%%%%%%%%%%%%%%%%%%%%%%%%
\subsection{Lessons}

The important lessons that we have learned include project planning,
collaboration skills, and iterative development. While we had prior
experience in all of these areas from previous projects, this project
helped ingrain these principles into how we worked, altogether adding to
a better workflow and increased productivity. 

For the project planning, we had thoroughly read through the
implementation section in \cite{shirvanian2d2fa}. From this, we 
broke down each element of the implementation into smaller chunks that
were easier to tackle and implement. This allowed us to develop one
module at a time, and ensure that module was functioning in the desired
way before continuing to the next module. 

In terms of collaboration, we have weekly meetings where we discuss what
we have accomplished in the past week, and what we plan on completing
for the next week. During the week, we update each other with our
progress, as well as asking questions or seeing if we have suggestions
for each other. 

With regard to iterative development, this goes hand in hand with our
project planning. Because we have broken down the problem into
bite-sized chunks, we can iteratively implement these small portions,
easily adding features and functionality to them as we progress, and
iteratively changing them or modifying them when we encounter the need
to do so.

%%%%%%%%%%%%%%%%%%%%%%%%%%%%%%%%%%%%%%%%%%%%%%%%%%%%%%%%%%%%
\subsection{Recommendations}


Possibly add to this with typing proof \cite{liuTypingProof}

%%%%%%%%%%%%%%%%%%%%%%%%%%%%%%%%%%%%%%%%%%%%%%%%%%%%%%%%%%%%
%%%%%%%%%%%%%%%%%%%%%%%%%%%%%%%%%%%%%%%%%%%%%%%%%%%%%%%%%%%%
\bibliographystyle{ieeetr}
\bibliography{ref}

\end{document}
