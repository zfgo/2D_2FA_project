\documentclass{article} % EXPAND THE PREAMBLE TO SEE ALL OF THE CODE
\usepackage[utf8]{inputenc}
\usepackage{amsmath, amsfonts} 
\allowdisplaybreaks
\usepackage{amssymb}
\usepackage{esint} % for more integral symbols
\usepackage[margin=1.5in]{geometry} % [margin=1.0in]
\usepackage{graphicx} % more arguments for the \includegraphics command
\usepackage{wrapfig} 
\usepackage{textcomp} % more text symbols
\usepackage{mwe} % used for idk?
\usepackage{gensymb} % for symbols (e.g \degree and \ohm)
\providecommand{\e}[1]{\ensuremath{\times 10^{#1}}}
\usepackage{multicol} % for columns
\usepackage{float} % to better format tables
\restylefloat{table} % to better format tables
\usepackage{tabularx} % to better format tables
\usepackage{titlesec} 
\usepackage{enumitem} % allows better formatting for list environments
\usepackage{tikz} % graphs and drawings
\newcommand{\comment}[1]{} % for multiline comments
\usepackage[stretch=10]{microtype} % best package ever
\usepackage{booktabs} % for all your fancy table needs
\usepackage{environ}
\usepackage{siunitx}
\usepackage{mathrsfs}
\usepackage{cancel}
\usepackage{multirow} % allows multirow cells in tables
\usepackage{xcolor} % fancier color options
\usepackage{tikz-qtree} % trees
\usepackage{forest} % more trees (better imo)
\usepackage{mathtools}
\usepackage{hyperref}

\usepackage{contour}
\usepackage{ulem}
\renewcommand{\ULdepth}{1.8pt}
\contourlength{0.8pt}

\newcommand{\myuline}[1]{
  \uline{\phantom{#1}}%
  \llap{\contour{white}{#1}}%
}



\renewcommand{\arraystretch}{1} 

\title{CS 433: 2D-2FA Project Midterm Report}
\author{Zane Globus-O'Harra, Doug Ure}
\date{\textit{12 May 2023}}

\begin{document}
\maketitle

\section{Introduction}

Two-factor authentication (2FA) is the contemporary approach to
authorizing a user. The first authentication factor is the user's
password, with the second factor being an additional piece of
information that only the user could know, often a PIN or a push
notification sent to the user over a secure channel to a trusted device. However, both
PIN-2FA and push-based 2FA have some issues, which are addressed in the
paper ``2D-2FA: A New Dimension In Two-Factor Authentication'' by
Maliheh Shirvanian and Shashank
Agrawal.\footnote{\href{https://arxiv.org/abs/2110.15872}{\texttt{https://arxiv.org/abs/2110.15872}}}

Some attacks to these two common 2FA methods include shoulder surfing,
short PINs, and neglectful user approvals. The authors present a new
approach to 2FA, which they have coined ``2D-2FA.'' In this new
approach, when a user logs in with their username and password, a
unique identifier is displayed to them. The user then inputs
this same identifier on their device. A one-time PIN is generated on the
device, and transferred automatically to the server, along with the
identifier. The identifier is used in the PIN's computation, so that the
PIN is bound to a specific session. 

The user's device and the server agree on a secret key during a one-time
registration process, which is also used in the PIN computation. Once
the PIN is transferred to the server, the server authenticates the
session associated with the identifier by verifying the PIN, thereby
taking two dimensions into account (the PIN and the identifier).

\subsection{Motivation}

We are motivated to work on this project for several key reasons.
Firstly, the subject of this project has real-world relevance. 2FA is
becoming increasingly more common to authenticate users, and as it
becomes more prevalent, attackers will focus more of their efforts on
finding ways to break through its layers of security. Our project will
help us learn about ways to further increase the security of 2FA by
using additional information along with the user's credentials and the
server-provided ``identifier.''

This is also a learning opportunity for us. Neither of us are very
familiar with security, and it is something that we are very interested
in learning about. By completing this project, we will develop valuable
technical skills and increase our knowledge base, as well as preparing
us for future projects and industry roles.

Lastly, this project could have a real impact on end-users. 2FA enhances
users' trust and confidence in online systems by making their personal
information and online interactions more secure. This project has the
potential to contribute to a larger goal of make a more secure digital
environment. 

\subsection{Objectives}

Our objectives have not changed much from when they were laid out in the
project proposal: we still plan to replicate the results of the 2D-2FA
paper using software. We will create programs for the server and the
device described in the 2D-2FA paper, requiring the programs to
communicate with each other, and implementing the 2D-2FA requirements.
To see specifically what we plan to deliver, see section 1.3.

\subsection{Deliverables}

Along with the final report, presentation, and poster, we plan on
completing the following software-related deliverables:
\begin{itemize}
    \item A working implementation of 2D-2FA.

    \begin{itemize}
        \item Programs for the server and the device, as described in
        the 2D-2FA paper. 

        \item This implementation will work across multiple devices.

        \item This implementation will work for multiple users.
    \end{itemize}

    \item Test cases for our code.

    \item Documentation.

    \begin{itemize}
        \item Installation instructions.

        \item Usage instructions.

        \item A design diagram.

        \item Well-commented code.
    \end{itemize}
\end{itemize}

Our implementation will use a 6-digit code as the identifier, instead of
a pattern or a QR code as suggested in the 2D-2FA paper. Additionally,
while we are planning for our implementation to work across multiple
devices, we plan on using devices that we know and trust, and do not
plan on deploying this software to arbitrary devices.

\section{Related Work}

Some related work and existing knowledge that we have used has mostly
been to understand the basics of the implementation of the 2D-2FA
presented in the paper. For example, when implementing the
device--server communications, Doug used online
resources\footnote{\href{https://realpython.com/python-sockets/}{\texttt{https://realpython.com/python-sockets/}}}
to help figure out these communications.

Zane worked on the hashing functionality, and for this he needed to
understand HMAC (hash-based message authentication code) to create a
hash of the identifier and the timestamp using the user's secret key as
the HMAC key. For this, he consulted the Wikipedia
page\footnote{\href{https://en.wikipedia.org/wiki/HMAC}{\texttt{https://en.wikipedia.org/wiki/HMAC}}}
to get a basic understanding, as well as Python's standard libraries for
hashing\footnote{\href{https://docs.python.org/3/library/hashlib.html}{\texttt{https://docs.python.org/3/library/hashlib.html}}}
and message
authentication.\footnote{\href{https://docs.python.org/3/library/hmac.html}{\texttt{https://docs.python.org/3/library/hmac.html}}}

Other than these outside sources, we have not needed to consult any
additional research cited by Shirvanian and Agrawal, nor have we needed
to find any further papers ourselves. Should we need to do additional
research as we continue our progress on this project, we will be sure to
include references to that material in our final report. 

\section{Current Progress}

So far, we have been meeting once a week to discuss our project, what we
have completed for the project since we last met, and what we plan on
completing before we meet again. We do not have a fixed timeline, but we
have a shared document where we have laid out the requirements for our
system, and each week we select a few items from this document to
complete. 

As of 2023-05-09, we have implemented a rough outline of the device and
the server such that they can communicate, as well as the basic 2D-2FA
functionality. Specifically, we have implemented the basic 2D-2FA
functionality of the device and the server. These two pieces of code
communicate, and a test identifier correctly grants authorization to a
test user with a fake secret key. 
% add some more details about what we have done

\section{Lessons}

The important lessons that we have learned, include project planning,
collaboration skills, and iterative development. For the project
planning, we have thoroughly read through the description of the
implementation of 2D-2FA written by Shirvanian and Agrawal. From this,
we have broken down each element of the implementation into smaller
chunks that are easier to tackle and carry out.

In terms of collaboration, we have weekly meetings where we discuss what
we have accomplished in the past week, and what we plan on completing
for the next week. During the week, we update each other with our
progress, as well as asking questions or seeing if we have suggestions
for each other. 

With regard to iterative development, this goes hand in hand with our
project planning. Because we have broken down the problem into
bite-sized chunks, we can iteratively implement these small portions,
easily adding features and functionality to them as we progress, and
iteratively changing them or modifying them when we encounter the need
to do so.

\section{Timeline and Future Plans}

Now that we have a base outline for our device and server programs that
allow them to communicate, as well as the basic 2D-2FA functionality,
our next objectives are to allow for some sort of system where users can
register, and our software can authenticate previously-registered users.
Next, we will implement functionality so that the device and the server
to be able to function across different machines. 

More specifically, our ``To-Do'' list contains the following items:
\begin{itemize}
    \item 2D-2FA functionality for multiple users.
    \item 2D-2FA functionality across multiple device.
    \item Create test cases.
    \item More in-depth documentation.
    \item Final report, presentation, and poster.
\end{itemize}
The items that we will be completing next will be first two on the above
list. 

% \section{References}

\end{document}
